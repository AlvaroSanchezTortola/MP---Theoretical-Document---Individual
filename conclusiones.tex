Los datos clasificados permitieron identificar a las variables más significativas. Las variables descriptivas de mayor significancia fueron: edad, escolaridad, empleo y acompañamiento psicológico. Respecto a las variables electroencefalográficas, el canal F3 fue el más representativo. Sin embargo, ésta no fue mucho mayor que los canales AF3 y AF4. 

Los kernels empleados para Máquina de Vectores de Soporte fueron: Lineal, Polinomial y Gaussiano RBF. El mejor desempeño lo tuvo el modelo con kernel Gaussiano RBF. Este modelo tuvo un desempeño de precisión del cincuenta y ocho (58) por ciento para detectar el engaño. Sus mejores variables $\gamma$ y $C$ fueron $2^{-9.63}$ y $2^3$ respectivamente.

La interfaz de usuario para el manejo del sistema de detección del engaño por parte de estudiantes de la facultad de Psicología fue descrita como "\textit{sencilla y fácil de entender}". Sin embargo, no resultó ser "\textit{intuitiva}" al momento de utilizarse por primera vez.

De esta manera, se logró elaborar una herramienta automatizada para apoyar a la detección del engaño empleando Máquina de Vectores de Soporte con kernel Gaussiano RBF. Sin embargo, este método no alcanzó el mejor desempeño de precisión del engaño, por lo que es recomendado emplear un modelo como \textit{K Vecinos Cercanos}.