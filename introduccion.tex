La tarea de los especialistas en la detección del engaño, actualmente es complicada, debido a la insuficiente integración de las diferentes técnicas, es decir que, éstas suelen trabajarse por separado lejos de lograr apoyarse unas con otras y así mejorar o conseguir un producto más certero. El uso de polígrafo para esta tarea suele ser la manera más recurrente. Sin embargo, este instrumento posee un bajo nivel de efectividad al momento de detectar el engaño (Wolpe, Foster, \& Langleben, 2010). Nuevas técnicas que emplean inteligencia artificial para solucionar este tipo de problemas apenas están saliendo a la luz, y su uso puede brindar una valiosa alternativa para abordar este problema.

Durante mucho tiempo, se ha sugerido la relación de las emociones humanas con patrones en la actividad cerebral. Nuevos instrumentos han permitido realizar innovadores estudios en búsqueda de respuestas a este tipo de inferencias. En efecto, temas como la naturaleza del engaño en los seres humanos ahora puede verse profundizada por estos nuevos métodos. La detección de mentiras forma parte de la vida cotidiana de muchas personas, y aun más, forma parte de la naturaleza de la vida del ser humano.

El empleo de nuevas técnicas para la detección del engaño se ha visto apoyado en el surgimiento de nuevas tecnologías en las últimas décadas. El poder computacional de los ordenadores actuales permiten realizar tareas exhaustivas en tiempos relativamente cortos. La inteligencia artificial y el aprendizaje de máquina se han vuelto una realidad en nuestro mundo tecnológico. Algoritmos como \textit{Máquina de Vectores de Soporte} apoyan ahora en tareas como la identificación de imágenes y el procesamiento de comandos de voz. En este trabajo se investiga si este tipo de algoritmos puede apoyar en la tarea de detección del engaño en el ser humano.