La necesidad de alternativas para la detección del engaño dentro del ámbito empresarial hace atractiva la idea de emplear inteligencia artificial. Los datos fisiológicos medidos por el polígrafo, por ejemplo, reflejan la actividad del sistema nervioso autónomo (SNA), por lo que reflejan no solo actividad inusual durante un engaño, sino también ansiedad en general, independientemente de la causa (Wolpe et al., 2010). El uso de análisis electroencefalográfico en los individuos de estudio es un método más directo, donde poco se ha explorado (Wolpe et al., 2010). Por lo que nuevos métodos de detección basados en esta metodología pueden ser de gran aporte, tanto para el ámbito empresarial, como para el campo de la inteligencia artificial en nuestro país.   