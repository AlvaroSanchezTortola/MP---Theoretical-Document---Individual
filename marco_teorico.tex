\section{La Naturaleza del Engaño}
La detección del engaño y la confirmación de la verdad con la poligrafía convencional planteó una serie de problemas técnicos y éticos (Wolpe et al., 2010).  El análisis de señales electromagnéticas del cerebro se ha postulado como una alternativa viable para la detección de la mentira en ámbitos legales y profesionales. Incluso algunos métodos se promueven como más efectivos que el polígrafo convencional. Sin embargo, estos métodos aun poseen limitaciones en cuanto a la falta de exploración de los mismos y a la poca estandarización que se tiene en su campo de estudio (Wolpe et al., 2010). 
\section{El Cerebro Humano y su Actividad}
En la actualidad, se ha eludido la relación entre la estructura del cerebro humano con funciones humanas. De igual manera, se elude la relación entre patrones de actividad neuronal con actividades del ser humano (Davatzikos et al., 2005). El análisis de patrones cuantitativos en el espacio-tiempo de la actividad cerebral conlleva a un análisis multivariable, poco explorado a principios de siglo por la limitación del procesamiento de datos de aquel entonces. El uso de algoritmos de Machine Learning para clasificar complejos patrones de activación cerebral se ha empezado a abordar en diversos estudios (Davatzikos et al., 2005). 
\section{Aprendizaje de Máquina}
Support Vector Machine (SVM) es un poderoso método de clasificación que encuentra la hiper-superficie que maximiza el margen entre dos distribuciones (Chih-Wei Hsu, Chih-Chung Chang, 2008), las respuestas verdaderas y no verdaderas en nuestro caso. El objetivo de un SVM es producir un modelo que predice los valores objetivo de un set de datos, a partir de únicamente los atributos del mismo set.  Formalmente, se describe como: dado un set de valores de la forma \gls{(x_i,y_i), i=1, …, l donde x_i∈R^n y y∈〖{1,-1}〗^l}, el SVM (Boser, Guyon, & Vapnik, 1992) requiere la solución al siguiente problema de optimización: 
\section{Máquina de Vectores de Soporte}
\subsection{Kernels Lineales y no Lineales}



