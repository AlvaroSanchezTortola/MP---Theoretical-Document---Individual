En la actualidad, los procedimientos para la detección del engaño en el ámbito empresarial son muy limitados.  En la mayoría de las empresas, cuando se necesita realizar estas pruebas, se recurre al polígrafo. Sin embargo, el uso de este dispositivo no lanza un resultado muy preciso. La necesidad de obtener resultados más exactos nos llevan a buscar recursos en las neurociencias que nos permitan alcanzarlos.  El desarrollo de la inteligencia artificial en los últimos años ha generado nuevas alternativas para afrontar esta problemática. En este módulo se obtendrá información procedente del análisis electroencefalográfico aplicado a varios individuos; con el objetivo de detectar si los resultados de los mismos nos muestran cambios significativos que nos lleven a comprobar si los mismos, están mintiendo. Se utilizará algoritmos basados en Máquina de Vectores de Soporte no lineal para el procesamiento de los datos electroencefalográficos, provenientes del dispositivo \textit{Emotiv Epoc}. Como resultado, se obtendrán las características de estas señales que mejor puedan detectar el engaño, se producirá el código de un programa de detección de engaño, el porcentaje de efectividad del mismo y la interfaz de uso.