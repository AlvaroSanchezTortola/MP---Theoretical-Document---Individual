En la actualidad, existen limitados procedimientos para la detección del engaño en el ámbito empresarial. El desarrollo de la inteligencia artificial en los últimos años está permitiendo nuevas alternativas para afrontar esta situación. En este módulo se obtendrá información procedente del análisis electroencefalográfico de varios individuos, con el objetivo de poder detectar si una persona está mintiendo. Se utilizará algoritmos basados en \textit{Máquina de Vectores de Soporte} no lineal para el procesamiento de los datos, provenientes del dispositivo \textit{Emotiv Epoc}. Como resultado, se obtendrá el código de un algoritmo de detección de engaño basado en, su documentación de uso y el porcentaje de efectividad del mismo. 