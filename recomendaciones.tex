En cuanto a la elección de características de medición para la detección del engaño, se recomienda experimentar y explorar otras características descriptivas que puedan brindar un perfil más específico de los sujetos de prueba. Características inusuales o que puedan segmentar de una mejor manera a la población objetivo, puede brindar un acercamiento distinto para la detección de sus emociones. Del mismo modo, escoger una población de estudio más amplia y general puede ayudar a representar de una mejor manera al ser humano como tal. Es decir, un mayor número de participantes y una mayor variabilidad en las características de estos puede ayudar a generalizar los elementos presentes en los seres humanos para la detección efectiva del engaño.

En cuanto a los modelos de predicción desarrollados, se recomienda utilizar el algoritmo \textit{K Vecinos Cercanos} para la predicción del engaño utilizando señales electroencefalográficas. Sin embargo, sería de mucho interés implementar combinaciones de algoritmos, como Máquina de Vectores de Soporte combinado con K Vecinos Cercanos, para obtener algoritmos más robustos y verificar si existe una mejora en la predicción del engaño de esta manera. Por otro lado, se recomienda realizar el procesamiento de los datos en un sistema operativo que permita a la librería \textit{TensorFlow} utilizar la tarjeta de vídeo externa para una mayor capacidad de procesamiento. Para eso, podría emplearse el sistema operativo \textit{Microsoft Windows} en conjunto con las tecnología \textit{CUDA de Nvidia}. 