La tarea de los especialistas en la detección del engaño, actualmente es complicada, debido a la insuficiente integración de las diferentes técnicas, es decir que, éstas suelen ser trabajadas por separado lejos de lograr apoyarse unas con otras y así mejorar o conseguir un producto más certero. El uso de polígrafo para esta tarea suele ser la manera más recurrente. Sin embargo, este instrumento posee un bajo porcentaje de efectividad al momento de detectar el engaño (Wolpe, Foster, \& Langleben, 2010). Nuevas técnicas que emplean inteligencia artificial para solucionar este tipo de problemas apenas están saliendo a la luz, y su uso puede brindar una valiosa alternativa para abordar este problema.

La atribución de patrones en la actividad cerebral relacionados a emociones humanas ha sido sostenida por investigadores por mucho tiempo. Herramientas innovadoras que permiten la lectura de esta actividad por medio de señales electroencefalográficas hacen tentador su uso para poder determinar el estado en el que se encuentra el pensamiento humano en distintas situaciones. Su aplicación para la detección del engaño ha sido, hasta hace poco, una idea que no había podido ser materializada. Estas herramientas pueden brindar nuevos avances en la tarea de detección del engaño y el surgimiento de nuevas técnicas para el mismo. 

El empleo de la inteligencia artificial en las actividades humanas ha tenido gran repercusión en la actualidad. El desarrollo de máquinas con gran poder de procesamiento han permitido que estas tecnologías se encuentren ahora al alcance de un mayor número de personas. Es por esto que nuevas investigaciones han encontrado finalmente desarrollo con técnicas que antes las hacían poco viables. La detección del engaño no es una excepción a esto. El empleo del aprendizaje de máquina para apoyar estos estudios puede ser clave para el surgimiento de metodologías más precisas y eficientes.  