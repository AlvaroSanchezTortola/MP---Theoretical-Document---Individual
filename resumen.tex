La tarea de los especialistas en la detección del engaño, actualmente es complicada, debido a la insuficiente integración de las diferentes técnicas, es decir que, éstas suelen ser trabajadas por separado lejos de lograr apoyarse unas con otras y así mejorar o conseguir un producto más certero. El uso de polígrafo para esta tarea suele ser la manera más recurrente. Sin embargo, este instrumento posee un bajo porcentaje de efectividad al momento de detectar el engaño (Wolpe, Foster, \& Langleben, 2010). Nuevas técnicas que emplean inteligencia artificial para solucionar este tipo de problemas apenas están saliendo a la luz, y su uso puede brindar una valiosa alternativa para abordar este problema.

En el siguiente trabajo, se realizó una serie de análisis para la obtención de las características más significativas relacionadas al engaño. Se desarrolló un programa basado en Máquina de Vectores de Soporte para predecir si un individuo se encuentra mintiendo o no. Se diseñó una interfaz de usuario que permite a una persona hacer uso de los programas desarrollados de manera amigable. De esta manera, se obtuvieron características como la edad y el estado de empleo, que pueden estar relacionadas al engaño. Se determinó el área cerebral en donde hay mayor actividad al momento del engaño, a través de los canales electroencefalográficos medidos más significativos. La Máquina de Vectores de Soporte basada en kernel Gaussiano RBF es el que mejor precisión tiene al predecir el engaño. Sin embargo, resultados de otros módulos del proyecto sugieren mejores algoritmos para esta tarea. 